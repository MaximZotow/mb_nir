\begin{center}
	\normalsize\bfseries\MakeUppercase{введение}
\end{center}
\addcontentsline{toc}{section}{ВВЕДЕНИЕ}

Требование точности к продукции выдвигает критерий повторяемости на первый план. Повторяемость может быть достигнута путём автоматизации техпроцесса, исключая человеческий фактор при его проведении. С этим может справиться дистанционное автоматизированное управление, которое решает следующие проблемы:

\begin{itemize}
	\item простой оборудования в связи с отсутствием оператора;
	\item уменьшение производительности в связи с необходимостью ручного управления;
	\item ухудшение качества продукции в связи с неточным следованиям инструкциям.
\end{itemize}


\textit{Целью работы является анализ современных протоколов передачи данных между оборудованием и системами человек-оборудование, а также выбор наиболее оптимальных для осуществления автоматизации вакуумных систем.}

\textbf{Задачи}:
\begin{itemize}
	\item  составить терминологический словарь и список ключевых слов; 
	\item  сформировать поисковые запросы; 
	\item  найти литературу по данной теме и провести анализ на основе полученных данных:
	\begin{itemize}
		\item проанализировать протоколы транспортного уровня;
		\item проанализировать протоколы промышленного интернета, узнать их достоинства и недостатки;
	\end{itemize}
	\item  провести краткий обзор информационных материалов;
	\item  систематизировать и обобщить информацию.
	\item сделать выводы о преимуществах автоматизации, предложить протокол, способный реализовать поставленную задачу.
\end{itemize}



