\newglossaryentry{connection_protocol}{name={протокол связи},description={набор определённых правил или соглашений интерфейса логического уровня, который определяет обмен данными между различными устройствами. Эти правила задают единообразный способ передачи сообщений и обработки ошибок}}
\newglossaryentry{PLK}{name={программируемый логический контроллер},description={специальная разновидность электронной вычислительной машины, чаще всего используется для автоматизации}}
\newglossaryentry{udalenka}{name={удалённый доступ},description={система, позволяющая пользователю подключаться к аппаратуре, не находясь в непосредственной близости к ней}}
\newglossaryentry{scada}{name={SCADA},description={\textbf{S}upervisory \textbf{C}ontrol \textbf{A}nd \textbf{D}ata \textbf{A}cquisition (диспетчерское управление и сбор данных) ---  программный пакет, предназначенный для разработки или обеспечения работы в реальном времени систем сбора, обработки, отображения и архивирования информации об объекте мониторинга или управления. Используется везде, где требуется обеспечить контроль оператора за \Gls{asu_tp}}}
\newglossaryentry{asu_tp}{name={АСУ ТП},description={\textbf{А}втоматизированная \textbf{с}истема \textbf{у}правления \textbf{т}ехнологическим \textbf{п}ро\-цессом --- группа решений технических и программных средств, предназначенных для автоматизации управления технологическим оборудованием на промышленных предприятиях}}
\newacronym{cps}{ЦПС}{\textbf{Ц}ифровые \textbf{п}ромышленные \textbf{с}ети}
\newglossaryentry{modbus}{name={ModBus},description={открытый коммуникационный протокол, основанный на архитектуре ведущий — ведомый (master-slave). Широко применяется в промышленности для организации связи между электронными устройствами}}
\newglossaryentry{profibus}{name={ProfiBus},description={\textbf{Pro}cess \textbf{Fi}eld \textbf{Bus} -- шина полевого уровня -- открытая промышленная сеть, прототип которой был разработан компанией Siemens AG для своих промышленных контроллеров Simatic. Очень широко распространена в Европе, особенно в машиностроении и управлении промышленным оборудованием}}
\newglossaryentry{fieldbus}{name={Foundation Fieldbus},description={открытая архитектура, является цифровой, последовательной, двусторонней системой связи, которая служит в качестве базового уровня сети в заводских или фабричных системах автоматизации. Более подробно см \refpar{par:ffbus}}}
\newacronym{pdu}{PDU}{\textbf{P}rotocol \textbf{D}ata \textbf{U}nit (элемент данных протокола)}
\newacronym{adu}{ADU}{\textbf{A}pplication \textbf{D}ata \textbf{U}nit (пакет \mb целиком с заголовками, PDU и контрольной суммой)}
\newacronym{mmi}{MMI}{\textbf{M}achine \textbf{M}an \textbf{I}nterface (интерфейс соединения человека и машины, например - экран монитора)}
\newacronym{rsu}{РСУ}{\textbf{Р}аспределённая \textbf{С}истема \textbf{У}правления}
\newacronym{plk}{ПЛК}{\textbf{П}рограммируемый \textbf{Л}огический \textbf{К}онтроллер}
\newacronym{osi}{OSI}{\textbf{O}pen \textbf{S}ystems \textbf{I}nterconnection model -- представляет собой концептуальную модель, которая характеризует и стандартизирует коммуникационные функции телекоммуникационной или вычислительной системы безотносительно к лежащей в ее основе внутренней структуре и технологии. Его цель -- взаимодействие различных систем связи со стандартными протоколами связи}
\newacronym{tcp}{TCP}{\textbf{T}ransmission \textbf{C}ontrol \textbf{P}rotocol (протокол управления передачей) -- один из самых популярных протоколов транспортного уровня}
\newacronym{udp}{UDP}{\textbf{U}ser \textbf{D}atagram \textbf{P}rotocol (протокол пользовательских датаграмм)}
\newglossaryentry{socket}{name={сокет},description={название программного интерфейса для обеспечения обмена данными между процессами. Процессы при таком обмене могут исполняться как на одной ЭВМ, так и на различных ЭВМ, связанных между собой сетью. Сокет — абстрактный объект, представляющий конечную точку соединения}}
\newacronym{ecn}{ECN}{\textbf{E}xplicit \textbf{C}ongestion \textbf{N}otification (уведомление о переполнении канала трафика без потери пакетов)}
\newacronym{urg}{URG}{\textbf{U}rgent \textbf{P}ointer \textbf{F}ield (поле срочности)}
\newacronym{ack}{ACK}{\textbf{Ack}nowledgement field (поле подтверждения)}
\newacronym{psh}{PSH}{Push function (функция отправки)}
\newacronym{rst}{RST}{\textbf{R}e\textbf{s}e\textbf{t} the connection (разрыв соединения)}
\newacronym{syn}{SYN}{\textbf{Syn}chronize sequence numbers (флаг, используемый для установки соединения)}
\newacronym{fin}{FIN}{\textbf{Fin}ished sending data (окончание приёма, окончание соединения)}
\newacronym{ece}{ECE}{\textbf{EC}N-\textbf{E}cho (сообщает о том, что получатель поддерживает \textit{ECN})}
\newglossaryentry{octet}{name={октет},description={восемь двоичных разрядов. В русском языке октет обычно называют байтом}}
\newacronym{ftp}{FTP}{\textbf{F}ile \textbf{T}ransfer \textbf{P}rotocol (протокол передачи данных)}
\newacronym{http}{HTTP}{\textbf{H}ypertext \textbf{T}ransfer \textbf{P}rotocol (протокол передачи гипертекста)}
\newacronym{imap}{IMAP}{\textbf{I}nteractive \textbf{M}ail \textbf{A}ccess \textbf{P}rotocol (интерактивный протокол доступа к почте)}
\newacronym{pop}{POP}{\textbf{P}ost \textbf{O}ffice \textbf{P}rotocol (почтовый протокол)}
\newacronym{rlogin}{RLogin}{\textbf{R}emote \textbf{Login} (удалённый доступ)}
\newacronym{smtp}{SMTP}{\textbf{S}imple \textbf{M}ail \textbf{T}ransfer \textbf{P}rotocol (простой почтовый протокол)}
\newacronym{ssh}{SSH}{\textbf{S}ecure \textbf{Sh}ell (безопасная оболочка)}
\newglossaryentry{datagram}{name={датаграмма},description={блок информации, передаваемый протоколом через сеть связи без предварительного установления соединения и создания виртуального канала. Протоколы, использующие датаграммы работают быстрее, но не гарантируют доставки сообщений получателю}}
\newglossaryentry{fbus}{name={fieldbus},description={термин, обозначающий соединение полевых устройств (датчики, сенсоры, ПЛК) с человеко - машинным интерфейсом}}
\newacronym{rtu}{RTU}{\textbf{R}emote \textbf{T}erminal \textbf{U}nit, устройство связи с объектом. Используется для ввода сигналов с объекта в автоматизированную систему и вывода сигналов на объект}
\newglossaryentry{h1}{name={H1},description={одна из реализаций протокла Foundation Fieldbus}}
\newacronym{hse}{HSE}{\textbf{H}igh \textbf{S}peed \textbf{E}nternet (высокоскоростной интернет). Одна из реализаций протокола Foundation Fieldbus}
\newglossaryentry{pole}{name={полевая шина},description={канал общения между контроллером и другими устройствами}}
\newglossaryentry{polevoy}{name={полевой датчик},description={под этим термином понимается некое устройство, расположенное на удалении от системы контроля и занимающееся сбором и передачей информации}}
\newglossaryentry{collision}{name={коллизия},description={в терминологии компьютерных и сетевых технологий наложение двух и более кадров от станций, пытающихся передать кадр в один и тот же момент времени в среде передачи коллективного доступа}}
\newacronym{mbap}{MBAP}{\textbf{M}odbus \textbf{A}pplication \textbf{P}rotocol (прикладной протокол Modbus)}
\newacronym{dp}{DP}{\textbf{D}ecentralized \textbf{P}eripherals (Децентрализованные периферийные устройства)}
\newacronym{pa}{PA}{\textbf{P}rocess \textbf{A}utomation (автоматизация процессов)}
\newacronym{fms}{FMS}{\textbf{F}ieldbus \textbf{M}essage \textbf{S}pecification (спецификация сообщений Fieldbus)}
\newglossaryentry{marker}{name={маркерный доступ},description={по сети перемещается небольшой блок данных, называемый маркер. Владение этим маркером гарантирует право передачи. Если узел, принимающий маркер, не имеет информации для отправки, он просто переправляет маркер к следующей конечной станции. Каждая станция может удерживать маркер в течение определённого максимального времени (по умолчанию — 10 мс). Используется в топологии кольцо (см. \refpar{par:topology})}}
\newglossaryentry{watchdog}{name={сторожевой таймер},description={аппаратно реализованная схема контроля над зависанием системы. Представляет собой таймер, который периодически сбрасывается контролируемой системой. Если сброса не произошло в течение некоторого интервала времени, происходит принудительная перезагрузка системы}}
\newglossaryentry{profisafe}{name={Profisafe},description={набор технологий для предотвращения несчастных случаев на производстве}}
\newacronym{srd}{SRD}{\textbf{S}end and \textbf{R}eceive \textbf{D}ata with acknowledge (приём и отправка данных с уведомлением)}
\newacronym{snd}{SND}{\textbf{S}end \textbf{D}ata with \textbf{N}o acknowledge (отправка данных без уведомления)}


\newacronym{sd}{SD}{\textbf{S}tart \textbf{D}elimeter (стартовый разделитель)}
\newacronym{le}{LE}{\textbf{Le}ngth of telegram (длина телеграммы)}
\newacronym{ler}{LEr}{\textbf{Le}ngth of telegram \textbf{r}eserved (длина телеграммы зарезервированная для защиты)}
\newacronym{da}{DA}{\textbf{D}estination \textbf{A}ddress (адрес получателя)}
\newacronym{sa}{SA}{\textbf{S}ource \textbf{A}ddress (адрес отправителя)}
\newacronym{fc}{FC}{\textbf{F}unction \textbf{C}ode (код функции)}
\newacronym{sap}{SAP}{\textbf{S}ervice \textbf{A}ccess \textbf{P}oints (сервисные точки доступа, команда)}
\newacronym{du}{DU}{\textbf{D}ata \textbf{U}nit (данные)}
\newacronym{ed}{ED}{\textbf{E}nd \textbf{D}elimeter (конечный разделитель)}
\newacronym{ddl}{DDL}{\textbf{D}evice \textbf{D}escription \textbf{L}anguage (язык описания устройств)}
\newacronym{dd}{DD}{\textbf{D}evice \textbf{D}escription (описание устройства)}
\newacronym{pak}{ПАК}{\textbf{П}рограммно - \textbf{А}ппаратный \textbf{К}омплекс}

\newglossaryentry{automation}{name={автоматизация}, description={машинное производство, обычно осуществляемое под контролем компьютера и не требующее непосредственного вмешательства человека. Автоматизация особенно полезна в тех случаях, когда необходима чрезвычайная точность, а также при работе с опасными материалами, когда обеспечение безопасности человека оказывается слишком трудным и дорогостоящим}}
