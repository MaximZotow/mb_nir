\documentclass[../AISTR.tex]{subfiles}
\begin{document}
\begin{center}
	\normalsize\bfseries\MakeUppercase{ВВЕДЕНИЕ}
\end{center}
\addcontentsline{toc}{section}{ВВЕДЕНИЕ}

Требование точности к продукции выдвигает критерий повторяемости на первый план. Повторяемость может быть достигнута путём автоматизации техпроцесса, исключая человеческий фактор при его проведении. С этим может справиться дистанционное автоматизированное управление, которое решает следующие проблемы:
\begin{itemize}
	\item простой оборудования в связи с отсутствием оператора;
	\item уменьшение производительности в связи с необходимостью ручного управления;
	\item ухудшение качества продукции в связи с неточным следованиям инструкциям.
\end{itemize}


\textit{Целью работы является изучение библиотеки Qt, протокола \mb и разработка ПО по моделированию поведения частиц методом Монте-Карло.}

\textbf{Задачи}:
\begin{itemize}
	\item  изучить выбранный по результатам научной работы протокол передачи данных \mb \tcp;
	\item освоить принцип работы в одной из современных сред разработки и программирования Qt;
	\item разработать программу обмена сигналами между объектами;
	\item разработать клиент-серверную модель передачи данных по протоколу TCP;
	\item разработать ПО для моделирования процесса нанесения плёнки на тонкостенную трубку по методу Монте-Карло; результаты моделирования будут испольваны как входные данные для дисциплины <<Техника эксперимента в электронике и наноэлектронике>>.
\end{itemize}

\end{document}