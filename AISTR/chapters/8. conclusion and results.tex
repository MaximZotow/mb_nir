\documentclass[../AISTR.tex]{subfiles}
\begin{document}
\begin{center}
	\normalsize\bfseries\MakeUppercase{выводы и результаты}
\end{center}
\addcontentsline{toc}{section}{ВЫВОДЫ И РЕЗУЛЬТАТЫ}
\begin{enumerate}
	\item Был изучен выбранный по результатам литературного обзора протокол передачи данных между автоматизируемым оборудованием;
	\item Была создана программа обмена сигналами между двумя объектами класса с использованием встроенного в библиотеки Qt функционала сигналов и слотов;
	\item Разработаны приложения клиента и сервера для обмена данными между устройствами или между человеком и оборудованием по сети \tcp;
	\item Написана программа для моделирования нанесения покрытий на внутреннюю часть тонкой трубки методом Монте-Карло с использованием библиотек проекта Qt;
	\item По результатам моделирования была составлена математическая модель отношения толщин наносимого покрытия;
	\item Математическая модель, полученная в результате анализа данных по методике, описанной в курсе <<Техника эксперимента в электронике и наноэлектронике>> была проверена на адекватность;
	\item Благодаря проведению моделирования было установлено, что угол наклона трубки и её диаметр на отношение толщин нанесенного покрытия.
\end{enumerate}


\end{document}