\begin{center}
	\normalsize\bfseries\MakeUppercase{выводы и результаты}
\end{center}
\addcontentsline{toc}{section}{ВЫВОДЫ И РЕЗУЛЬТАТЫ}

\begin{enumerate}
	\item Был составлен терминологический словарь, сформированы поисковые запросы и найдена литература по промышленным протоколам и автоматизации;
	\item В результате проведённого литературного обзора были изучены два самых популярных сетевых протокола обмена данными: 
	\begin{itemize}
		\item TCP;
		\item UDP.
	\end{itemize}
Был выбран протокол \tcp, поскольку он:
\begin{itemize}
	\item ориентирован на качество соединения, а не на скорость;
	\item проверяет пакеты на целостность;
	\item имеет задержку передачи данных не сильно ниже, чем у \textit{UDP}, который славится своей быстротой;
	\item не нагружает канал, поскольку отправляет новый пакет только после подтверждения получения предыдущего.
\end{itemize}
	\item Были проанализированы три протокола промышленных сетей: 
	\begin{itemize}
	    \item \mb;
	    \item \pb;
	    \item \ffb.
	\end{itemize}
По результатам анализа всех трёх протоколов был выбран \mb, поскольку он:
\begin{itemize}
	\item прост в освоении;
	\item проверен временем и поддерживается большим числом производителей;
	\item является полностью открытым стандартом (в отличие от \pb);
	\item является наиболее распространённым (в странах СНГ протокол \textit{Foun\-dation Fieldbus}, несмотря на свои преимущества, не используется).
\end{itemize}
	\item Сделаны выводы о преимуществах автоматизации:
	\begin{itemize}
		\item уменьшение количества ошибок оператора;
		\item упрощение контроля за процессом;
		\item повышение надёжности процесса.
	\end{itemize}
	\item Была начата разработка собственной системы управление и контроля \newline SCADA.
\end{enumerate}
\newpage

\begin{center}
	\normalsize\bfseries\MakeUppercase{заключение}
\end{center}
\addcontentsline{toc}{section}{ЗАКЛЮЧЕНИЕ}

Технологии плотно вошли во все сферы нашей жизни. Не обходится без них и промышленность: благодаря наработкам \textit{IT}-индустрии мы имеем огромное количество протоколов обмена данными между устройствами и большое число промышленных протоколов, среди которых можно выделить три самых популярных:
\begin{enumerate}
	\item \mb;
	\item \pb;
	\item \ffb. 
\end{enumerate}

Промышленные протоколы используются при построении SCADA-систем для контроля за процессом производства. Эти системы помогают избежать таких недостатков ручного управления, как:
\begin{itemize}
	\item простой оборудования в связи с отсутствием оператора;
	\item уменьшение производительности в связи с необходимостью ручного управления;
	\item ухудшение качества продукции в связи с неточным следованиям инструкциям.
\end{itemize}

По результатам работы, проделанной в дисциплине <<Научно-исследовательс\-кая работа>>, была начата разработка собственной SCADA-системы при помощи библиотек Qt.
