\subsubsection{Foundation Fieldbus}\label{par:ffbus}
Один из самых молодых протоколов, появился в $1995$ как результат консорциума крупных производителей в ответ на задержки в разработке единого стандарта ``полевой шины'' \cite{galloway_introduction_2012}.

Многим этот протокол схож с \pb \textit{PA} \cite{__2002} :
\begin{itemize}
	\item работа во взрывоопасных зонах;
	\item передача сигнала вместе с питанием.
\end{itemize}
\ffb{} -- двухуровневый сетевой протокол, который \cite{__2001}:
\begin{itemize}
	\item объединяет компьютеры верхнего уровня;
	\item объединяет датчики, контроллеры, исполнительные механизмы.
\end{itemize}

К самым главным \textbf{преимуществам} стандарта можно отнести \cite{__2002, noauthor_foundation_2001}:
\begin{itemize}
	\item \ffb{} -- открытый протокол, для связи и создания приложений контроля;
	\item протокол делает ставку на распределённый интеллект, а не на центральный аппарат.  \ffb ориентирован на обеспечение одноранговой связи между узлами без центрального ведущего устройства. Этот подход даёт возможность реализовать системы управления, распределенные не только физически, но и логически, что во многих случаях позволяет повысить надежность и живучесть автоматизированной системы;
	\item все устройства, работающие по протоколу \ffb, должны работать между собой вне зависимости от производителя без дополнительных ухищрений;
	\item существует специальный язык описания оконечных устройств (DDL), позволяющий легко расширять сеть. Достаточно физически подключить новое устройство, и оно	тут же самоопределится на основании заложенного описания DD, после чего все функциональные возможности нового узла становятся доступными в сети.
\end{itemize}
Однако, у протокола есть и существенный недостаток:
\begin{itemize}
	\item малая распространённость в России, где конкурируют только \mb и \pb. 
\end{itemize}

