\usepackage{cmap}                                 % поиск в PDF
\makeatletter
\renewcommand{\paragraph}{\@startsection{paragraph}{4}{0ex}%
	{-3.25ex plus -1ex minus -0.2ex}%
	{1.5ex plus 0.2ex}%
	{\normalfont\normalsize\bfseries}}
\makeatother

\makeatletter
\renewcommand{\subparagraph}{\@startsection{subparagraph}{4}{0ex}%
	{-3.25ex plus -1ex minus -0.2ex}%
	{1.5ex plus 0.2ex}%
	{\normalfont\normalsize\bfseries}}
\makeatother
\setcounter{tocdepth}{5}
\setcounter{secnumdepth}{5}
\usepackage{mathtext}                           % русские буквы в формулах
\usepackage{fontspec}
\setmainfont{Times New Roman}                      % кодировка
\usepackage[utf8]{inputenc}                     % кодировка исходного текста
\usepackage[english,russian]{babel}     % локализация и переносы
\usepackage[left=2cm, right=1cm, top=2cm, bottom=2cm]{geometry}
\linespread{1.5}
\usepackage{titlesec, lipsum}
\setlength{\parindent}{1.25cm}
%\usepackage{fancyhdr}
%\pagestyle{fancy}
%\fancyhf{}
%\fancyhead[R]{\thepage}
\addto\captionsrussian{\renewcommand{\contentsname}{Содержание}}
\usepackage{indentfirst}
\usepackage{array}
\newcolumntype{C}[1]{>{\centering\arraybackslash}m{#1}}
\newcommand{\razm}[1]{\hspace{1ex} \ensuremath{\left[\text{#1}\right]}}
\usepackage{longtable}
\usepackage{multirow}
\usepackage{amsmath,amsfonts,amssymb,amsthm,mathtools}
\usepackage{mathtext} 
\usepackage{mathrsfs}
\usepackage[labelsep=period]{caption}
\usepackage{hyperref}\mathtoolsset{showonlyrefs=true} 
\hypersetup{
    colorlinks,
    citecolor=black,
    filecolor=black,
    linkcolor=black,
    urlcolor=blue
}
\usepackage{pgfplots}
\pgfplotsset{compat=1.15}
\usepackage{mathrsfs}
\usetikzlibrary{arrows}
\renewcommand{\phi}{\varphi}
\renewcommand{\epsilon}{\varepsilon}
\renewcommand{\abstractname}{Реферат}
\newcommand{\grad}{\ensuremath{^\circ}}
\usepackage{mathrsfs}
\usepackage{tocloft}
\addtolength{\cftsubsecnumwidth}{20pt}
%%% Дополнительная работа с математикой
\usepackage{amsmath,amsfonts,amssymb,amsthm,mathtools} % AMS
\usepackage{icomma} % "Умная" запятая: $0,2$ ---- число, $0, 2$ ---- перечисление
%% Номера формул
\mathtoolsset{showonlyrefs=false} % Показывать номера только у тех формул, на которые есть \eqref{} в тексте.
%\usepackage{leqno} % Нумерация формул слева
%%% Работа с таблицами
\usepackage{array,tabularx,tabulary,booktabs} % Дополнительная работа с таблицами
\usepackage{longtable}  % Длинные таблицы
\usepackage{multirow} % Слияние строк в таблице

\usepackage{float} 


%\numberwithin{figure}{section}
\usepackage{graphicx}
\graphicspath{{pics/}}
\newcommand{\rbf}[1]{\textbf{\ref{#1}}}
\newcommand{\ris}[1]{(рис. \rbf{#1})}
\newcommand{\tab}[1]{(табл. \rbf{#1})}
\newcommand{\rex}[2]{%
\begin{equation}
#1,
\end{equation}
где
\begin{itemize}
 {#2}
\end{itemize}%
}
\renewcommand{\frac}{\dfrac}
\usepackage{lastpage}
\usepackage[american, europeanresistors,americaninductors]{circuitikz}
\newcommand{\whr}{\text{, где}}
\usepackage{tabularx}
\usepackage{rotating}
\newcommand{\rom}[1]{%
  \textup{\textit{\uppercase\expandafter{\romannumeral#1}}}%
}
\newenvironment{aleq}{\begin{equation}\begin{aligned}}{\end{aligned}\end{equation}}
\usepackage{multicol}

\newcommand{\I}[2][\bullet]{\ensuremath{\accentset{#1}{I}_{#2}}}
\newcommand{\UU}[2][\bullet]{\ensuremath{\accentset{#1}{U}_{#2}}}
\newcommand{\R}[1]{\ensuremath{R_{#1}}}
\newcommand{\z}[1]{\ensuremath{\underline{z_{#1}}}}
\newcommand{\XL}[1]{\ensuremath{X_{L_{#1}}}}
\newcommand{\XC}[1]{\ensuremath{X_{C_{#1}}}}
\newcommand{\E}[1]{\ensuremath{E_{#1}}}
\newcommand{\A}{\razm{А}}
\newcommand{\V}{\razm{В}}
\newcommand{\WT}{\razm{Вт}}
\newcommand{\WAR}{\razm{Вар}}
\newcommand{\Om}{\razm{Ом}}
\newcommand{\frc}[1]{\frac{1}{#1}}
\newcommand{\rt}[1]{_\text{#1}}
\newcommand{\fii}[1]{\phi_{#1}}


\usepackage{amssymb, amsmath}
\usepackage{accents} 




\usepackage{listings}
\usepackage{color} %red, green, blue, yellow, cyan, magenta, black, white
\definecolor{mygreen}{RGB}{28,172,0} % color values Red, Green, Blue
\definecolor{mylilas}{RGB}{170,55,241}	
	
	\lstset{language=Matlab,%
		%basicstyle=\color{red},
		breaklines=true,%
		morekeywords={matlab2tikz},
		keywordstyle=\color{blue},%
		morekeywords=[2]{1}, keywordstyle=[2]{\color{black}},
		identifierstyle=\color{black},%
		stringstyle=\color{mylilas},
		commentstyle=\color{mygreen},%
		showstringspaces=false,%without this there will be a symbol in the places where there is a space
		numbers=left,%
		numberstyle={\tiny \color{black}},% size of the numbers
		numbersep=9pt, % this defines how far the numbers are from the text
		emph=[1]{for,end,break},emphstyle=[1]\color{red}, %some words to emphasise
		%emph=[2]{word1,word2}, emphstyle=[2]{style},    
	}

\newcommand{\Iarc}[1]{\arctg\left(\frac{\Im(\I{#1})}{Re(\I{#1})}\right)}


\usepackage{enumitem}
\renewcommand{\labelenumii}{\theenumii}

\renewcommand{\theenumii}{\theenumi.\arabic{enumii}.}
\newcolumntype{C}[1]{>{\centering\arraybackslash}m{#1}}
\usepackage{lscape}