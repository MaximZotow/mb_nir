\section{Обоснование и выбор области и средств поиска}
\subsection{Планируемые результаты поиска}
Найденая литература должна отвечать следующим требованиям:
\begin{itemize}
	\item \textbf{Актуальность} -- публикации не ранее 1998 (статьи выбраны с таким разбросом, поскольку некоторые протоколы на рынке практически полвека, в связи с чем новизна для некоторых из цитируемых статей не требуется)
	\item \textbf{Достоверность} -- публикации, прошедшие рецензирование и опубликованные на соответствующих для этого ресурсах (научные журналы или модерируемый компетентным сообществом сайт);
	\item \textbf{Соответствие запросу} -- содержание публикации соответствует направлению исследования.
\end{itemize}
\subsection{Области и средства поиска}
\begin{enumerate}
	\item \textbf{Научные статьи, патенты, диссертации} -- наиболее обширный раздел поиска. Даты написания варьируются	от 1998 до 2021 года. Весь материал изложен кратко и понятно, с сохранением	основной сути статьи и цели поиска.
	\item \textbf{Книги} -- много полезной информации, однако поиск затруднён в связи с обилием информации, не представляющей интереса для исследования.
	\item \textbf{Материалы из проверенных онлайн - ресурсов} -- материалы из таких источников, как \textit{habr.com} и ему подобные, поскольку статья перед публикацией проходит тщательный отбор модераторами и сообществом.
\end{enumerate}

Поиск источников осуществлялся с помощью поисковых систем \textit{Google, Google Scholar и ScienceDirect}, так
как данные системы ориентированы на поиск научных ресурсов больше, чем другие, что позволило быстро найти необходимую информацию и источники.
\pagebreak
\section{Краткий обзор информационных материалов}
\subsection{Оценка качества материалов, их достоверности}
\begin{center}
	\begin{longtable}{|C{0.13\linewidth}|C{0.35\linewidth}|C{0.15\linewidth}|C{0.1\linewidth}|C{0.15\linewidth}|}
		\caption{Результаты поиска по запросам}
		\label{tab:search_results}\\
		\hline
		\bfseries  Область поиска, система &\bfseries Наименование материала на языке оригинала, ссылка на источник & \bfseries Оценка достоверности & \bfseries Объём, кол-во страниц & \bfseries Номер позиции в списке литературы\\
		\endfirsthead
		\cpt\\
		\hline
		  Область поиска, система & Наименование материала на языке оригинала, ссылка на источник &  Оценка достоверности &  Объём, кол-во страниц &  Номер позиции в списке литературы\\
		\endhead
		\hline
		\gs &\multicite{hussein_wheeb_performance_2015}&\citefield{hussein_wheeb_performance_2015}{journaltitle}&\citefield{hussein_wheeb_performance_2015}{pages}&\cite{hussein_wheeb_performance_2015}\\
		\hline
		\gs &\multicite{kumar_survey_2012}&\citefield{kumar_survey_2012}{journaltitle}&\citefield{kumar_survey_2012}{pages}&\cite{kumar_survey_2012}\\
		\hline
		\gs &\multicite{noergaard_chapter_2010}&Elsevier Inc.&\citefield{noergaard_chapter_2010}{pages}&\cite{noergaard_chapter_2010}\\
		\hline
		\gs &\multicite{__2017-1}&Информа\-ционные технологии и телекоммуникации&\citefield{__2017-1}{pages}&\cite{__2017-1}\\
		\hline
		\gs &\multicite{__2016}&ООО <<Научно - техничес\-кий центр МЗТА>>&\citefield{__2016}{pages}&\cite{__2016}\\
		\hline
		\gs &\multicite{__2001}&\citefield{__2001}{journaltitle}&\citefield{__2001}{pages}&\cite{__2001}\\
		\hline
		\gs &\multicite{__2002}&Современ\-ные технологии автоматизации&\citefield{__2002}{pages}&\cite{__2002}\\
		\hline
		\gs &\multicite{__2018-1}&\citefield{__2018-1}{booktitle}&\citefield{__2018-1}{pages}&\cite{__2018-1}\\
		\hline
		\gs &\multicite{powell_profibus_2013}&\citefield{powell_profibus_2013}{journaltitle}&\citefield{powell_profibus_2013}{pages}&\cite{powell_profibus_2013}\\
		\hline
		\gs &\multicite{van_gorp_advanced_2009}&\citefield{van_gorp_advanced_2009}{journaltitle}&\citefield{van_gorp_advanced_2009}{pages}&\cite{van_gorp_advanced_2009}\\
		\hline
		\gs &\multicite{_modbus_2021}&\citelist{_modbus_2021}{publisher}&\citefield{_modbus_2021}{pages}&\cite{_modbus_2021}\\
		\hline
		\g &\multicite{advantech__2019}&Advantech IoT (\textit{habr.com})&1&\cite{advantech__2019}\\
		\hline
		\g &\multicite{phoenix_contact__2020}& Phoenix Contact (\textit{habr.com})&1&\cite{phoenix_contact__2020}\\
		\hline
		\g &\multicite{promwad__2019}& Promwad (\textit{habr.com})&1&\cite{promwad__2019}\\
		\hline
		\gs &\multicite{__2010}&Современ\-ные технологии автоматизации&\citefield{__2010}{pages}&\cite{__2010}\\
		\hline
		\gs &\multicite{daneels_what_1999}&\citefield{daneels_what_1999}{eventtitle}&\citefield{daneels_what_1999}{pages}&\cite{daneels_what_1999}\\
		\hline
		\gs &\multicite{__2019}&филиал ФГБОУ ВО РГУПС г. Воронеж&\citefield{__2019}{pages}&\cite{__2019}\\
		\hline
		\gs &\multicite{__2013-1}&Автомати\-зация и управление в технических системах&\citefield{__2013-1}{pages}&\cite{__2013-1}\\
		\hline
		\gs &\multicite{__1998}&Современ\-ные технологии автоматизации&\citefield{__1998}{pages}&\cite{__1998}\\
		\hline
		\gs &\multicite{__2013}&\citefield{__2013}{journaltitle}&\citefield{__2013}{pages}&\cite{__2013}\\
		\hline
		\gs &\multicite{__2017}&\citefield{__2017}{booktitle}&\citefield{__2017}{pages}&\cite{__2017}\\
		\hline
		\gs &\multicite{galloway_introduction_2012}&IEEE Communi\-cation surveys \& tutorials&\citefield{galloway_introduction_2012}{pages}&\cite{galloway_introduction_2012}\\
		\hline
		\gs &\multicite{__2020}&Издательст\-во Уральского университета&\citefield{__2020}{pagetotal}&\cite{__2020}\\
		\hline
		\gs &\multicite{a_design_2020}&\citefield{a_design_2020}{journaltitle}&\citefield{a_design_2020}{pages}&\cite{a_design_2020}\\
		\hline
		\gs &\multicite{swales_open_1999}&Modicon& 25&\cite{swales_open_1999}\\
		\hline
		\gs &\multicite{thomesse_fieldbus_2005}&\citefield{thomesse_fieldbus_2005}{journaltitle}&\citefield{thomesse_fieldbus_2005}{pages}&\cite{thomesse_fieldbus_2005}\\
		\hline
		\g &\multicite{promwad__2019-1}& Phoenix Contact (\textit{habr.com})&1&\cite{promwad__2019-1}\\
		\hline 
		\g &\multicite{__2015}&д.т.н. Виктор Денисенко&1&\cite{__2015}\\
		\hline
		\g &\multicite{acromag_introduction_2002}&Acromag Inc.&1&\cite{acromag_introduction_2002}\\
		\hline
		\gs &\multicite{vincent_foundation_2001}&\citefield{vincent_foundation_2001}{journaltitle}&\citefield{vincent_foundation_2001}{pages}&\cite{vincent_foundation_2001}\\
		\hline
		\gs &\multicite{noauthor_foundation_2001}&FIeldbus Inc.&36&\cite{noauthor_foundation_2001}\\
		\hline
		\gs &\multicite{_foundation_1999}&Современ\-ные технологии автоматизации&\citefield{_foundation_1999}{pages}&\cite{_foundation_1999}\\
		\hline
	\end{longtable}
\end{center}

\subsection{Краткое резюме по каждому источнику информации. Основные результаты, полученные авторами}
\begin{itemize}[label=]
	\item \cite{hussein_wheeb_performance_2015} ---  в данной работе проведён анализ протоколов, используемых для передачи данных в сети. На практике проанализированы два протокола обмена данных \textit{установка -- компьютер}, приведеных их достоинства и недостатки при помощи диаграмм и таблиц. Сделаны выводы о возможностях применения этих протоколов в различных ситуациях. 
	\item \cite{kumar_survey_2012} --- в статье проведён подробный разбор двух популярных протоколов связи. Цель работы -- ознакомить читателя с ключевыми терминами и в теории разъяснить разницу между двумя протоколами. Приведены достоинства и недостатки, а также границы применения каждого из них.
	\item \cite{noergaard_chapter_2010} --- в этой главе книги сжато и понятно показывается разница между двуми доминирующими транспортными протоколами: \textit{TCP} и \textit{UDP}. 
	\item \cite{__2017-1} --- статья описывает различные стандарты и протоколы для связи и автоматизации. Статья включает в себя описание протоколов прикладного, транспортного, сетевого и канального уровней модели \textit{TCP/IP}. Данная обзорная статья способствует более корректному выбору протоколов при построении сети, обнаружении и управлении устройствами, налаживания взаимодействия между объектами автоматизации.
	\item \cite{__2016} --- приводится обзор широко используемых и перспективных протоколов передачи данных приборов учета, используемых в России и Европе \textit{(Modbus,  DLMS/COSEM, M-BUS, IEC 61344, Euridis)}. В обзоре рассматриваются протоколы, отвечающие российским/европейским стандартам, и не рассматриваются частные фирменные разработки, из-за	их ограниченной сферы применения.
	\item \cite{__2001} --- в данной статье автор провёл анализ наиболее часто применяемых протоколов связи для автоматизации в современном мире. На примере популярных стандартных решений рассмотрена проблема практического применения промышленных сетевых технологий. Приведены виды топологий сети \textit{(общая шина, кольцо, звезда)}. Проведён анализ протоколов \textit{Foundation Fieldbus} и \textit{Profibus}.
	\item \cite{__2002} --- в статье рассматриваются вопросы построения распределённых систем \Gls{asu_tp} на базе современных сетевых решений. Приведены наиболее популярные \Gls{cps}: \textit{can, interbus, profibus, fieldbus}. Рассмотрены варианты объединения частей сети, работающей по разным протоколам в одну. 
	\item \cite{__2018-1} --- в статье был проведён краткий разбор сетевых технологий промышленного интернета: \textit{Modbus, EtherCAT, PROFINET, Powerlink} с целью ознакомления с различными технологиями. 
	\item \cite{powell_profibus_2013} --- в статье сравниваются два самых популярных в настоящее время протокола для автоматизации: \textit{Modbus} и \textit{Profibus}. Описаны сильные и слабые стороны технологий. Рассмотрены применения протоколов вместе и по отдельности. Сделаны выводы о сферах использования. Статья написана разработчиком стандарта \pb и может несколько принижать \mb.
	\item \cite{van_gorp_advanced_2009} --- статья сравнивает возможности таких протоколов, как \textit{Modbus, Profibus и пр}. Сравнение выполнено в виде списков с особенностями, преимуществами и недостатками. 
	\item \cite{_modbus_2021} --- в статье рассмотрен самый распространённый не только в мире, но и в России протокол \textit{Modbus RTU}. Приведены расшифровки аббревиатур для упрощения понимания. Описан формат передаваемого в сети пакета. Указаны достоинства и недостатки стандарта.
	Приведены его характеристики.
	\item \cite{advantech__2019} --- в статье разобрана реализация протокола \textit{Modbus}, форматы данных, программное обеспечение для работы с протоколом. Приведено подробное описание работы протокола, а также примеры применения.
	\item \cite{phoenix_contact__2020} --- в статье рассказывается о системах автоматизации на основе протокола \textit{Foundation Fieldbus}. Описаны возможности технологии, приведены составляющие компоненты сети. Описана топология, параметры сети и назначение каждого из компонентов системы. 
	\item \cite{promwad__2019} --- в статье подробно описана работа автоматизированных объектов на аппаратной и программной части. Рассмотрены способы сбора данных и подачи команд. Проведён разбор нижнего и верхнего уровней. Приведены виды проверенных и современных протоколов. 
	\item \cite{__2010} --- в статье описывается протокол \textit{Modbus}. Объяснена популярность на территории России. Описан формат передаваемого с помощью протокола пакета. Приведены отличия \textit{Modbus TCP} от \textit{Modbus RTU}. Указаны достоинства и недостатки протокола. 
	\item \cite{daneels_what_1999} --- в статье расшифровывается термин \textit{SCADA}, аппаратная и программная архитектура, способы связи между устройствами. Приведены данные по возможности масштабирования такой системы. Приведены стадии разработки программного обеспечения для автоматизации. Сделаны выводы о преимуществах таких систем с оговоркой на правильный инжиниринг.
	\item \cite{__2019} --- в статье разъяснён термин \textit{SCADA}. Приведены компоненты, входящие в SCADA - систему. Выявлена целевая функция таких систем. Выполнен анализ применения систем диспетчерского управления и сбора данных для автоматизации управления техническими процессами. Выделены структурные компоненты таких систем. Приведены примеры \newline SCADA - систем. 
	\item \cite{__2013-1} --- в статье приведён набор функций, повторяющийся во всех проектах по автоматизации. Приведены достоинства внедрения SCADA - систем. 
	\item \cite{__1998} --- в статье приводятся основные функции, которые должна выполнять SCADA - система. Приводятся достоинства и недостатки совмещения функций автоматического управления и операторского интерфейса. Приведены качества, которые должны быть присущи системе.
	\item \cite{__2013} ---  в данной статье рассматриваются основы протокола Modbus и его базовые принципы работы, а также приводится конкретный пример работы с ним. Приводятся преимущества и недостатки протокола. Приведённые в статье таблицы упрощают понимание материала. 
	\item \cite{__2017} --- авторы статьи на примере похожей установки производит автоматизацию по протоколу \textit{Modbus TCP}.
	\item \cite {galloway_introduction_2012} ---  данная статья повествует небольшую историю об индустриальных сетях. Приводятся популярные протоколы индустриальных сетей. Цель статьи -- послужить вводом в промышленные сети и сравнить их с обычными.
	\item \cite{__2020} --- учебное пособие УрФУ охватывает не только общетехнические аспекты сетей передачи данных, но и сконцентрировано на темах применения сетей в промышленной автоматизации технических систем.
	\item \cite{a_design_2020} --- в статье проектируется устройство, которое способно собирать любые данные и передавать их в сеть. Пример реализации протокола \textit{Mod\-bus};
	\item \cite{swales_open_1999} --- в данном материале приводится спецификация протокола \mb, основные команды и принципы работы с ним. Документ составлен компанией Modicon, которая ответственна за разработку \mb.
	\item \cite{thomesse_fieldbus_2005} --- в данной статье прослеживается история зарождения Fieldbus, принципиальные стадии развития. Проанализировано и классифицировано различие протоколов.
	\item \cite{promwad__2019-1} --- в статье рассмотрено, какие протоколы используются в системах по всему миру. Рассмотрен протокол \pb.
	\item \cite{__2015} --- энциклопедия инженера автоматизации, изданная д.т.н. Виктором Денисенко (автор одной из цитируемых статей \cite{__2010});
	\item \cite{acromag_introduction_2002} --- документация по \pb от одного из лидеров рынка промышленной автоматизации и интернета вещей.
	\item \cite{vincent_foundation_2001} --- описан стандарт \ffb и два его подстандарта H1 и HSE. Приведено описание физического уровня коммуникации (в соответствии со стандартом \osi). В деталях рассказано про архитектуру протокола.
	\item \cite{noauthor_foundation_2001} --- документация стандарта \textit{Foundation Fieldbus}, созданная разработчиком этого протокола.
	\item \cite{_foundation_1999} --- в статье приводится сранение двух современных протокола автоматизации: \pb и \ffb. Приведены сходства и различия, достонства и недостатки. Сделан вывод о выборе наилучшего протокола.
\end{itemize}