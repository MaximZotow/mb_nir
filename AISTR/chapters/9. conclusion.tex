\documentclass[../AISTR.tex]{subfiles}
\begin{document}
\begin{center}
\normalsize\bfseries\MakeUppercase{заключение}
\end{center}
\addcontentsline{toc}{section}{ЗАКЛЮЧЕНИЕ}

По результатам работы были изучены различные протоколы, используемые для связи оборудования в сетях, а также промышленные протоколы.

Было выбрано наиболее оптимальное сочетание протоколов: \mb и \tcp.

Были разработаны программы для отправки и принятия сигналов и данных от различного оборудования, а также проведено моделирование нанесения покрытия на внутреннюю часть тонких трубок по методу Монте-Карло.
	
Дальнейшим направлением работы является усовершенствование программы обмена данными между устройствами и осуществление автоматизации оборудования по протоколу \mb \tcp, поскольку дистанционное автоматизированное управление позволяет минимизировать такие проблемы, как:
\begin{itemize}
	\item простой оборудования в связи с отсутствием оператора;
	\item уменьшение производительности в связи с необходимостью ручного управления;
	\item ухудшение качества продукции в связи с неточным следованиям инструкциям.
\end{itemize}


\end{document}